%!TEX program = xelatex
\documentclass[UTF8,AutoFakeBold=1,AutoFakeSlant,zihao=-4,twoside]{cucthesis}

\usepackage[UTF8]{ctex}  
\usepackage{enumerate}
\usepackage{enumitem}

% \usepackage{algpseudocode}
\usepackage[normalem]{ulem}
\usepackage{subcaption}
\usepackage{multirow}
\usepackage{hhline}
% \usepackage{graphicx}
% \usepackage{caption}
\usepackage{booktabs}
\usepackage{makecell}
\usepackage[justification=centering]{caption}

% 算法包,可以使用算法块
\usepackage[ruled,vlined]{algorithm2e}
\renewcommand{\algorithmcfname}{算法 -}  %<---细节与重点
\SetKwInput{KwIn}{输入}  %<---细节与重点
\SetKwInput{KwOut}{输出}  %<---细节与重点

\usepackage{amssymb}


\usepackage{graphicx}
\usepackage{subcaption}

% 控制table的模块
\usepackage{tabularx}
\usepackage{array}
\usepackage{booktabs} %调整表格线与上下内容的间隔
\usepackage{multirow}
\usepackage{diagbox}



\usepackage{pifont}
\usepackage[perpage,symbol*]{footmisc}
\DefineFNsymbols{circled}{{\ding{192}}{\ding{193}}{\ding{194}}
{\ding{195}}{\ding{196}}{\ding{197}}{\ding{198}}{\ding{199}}{\ding{200}}{\ding{201}}}
\setfnsymbol{circled}


% 在这里填写你的论文题目
\newcommand{\myTitleCH}{多频点数字上变频器设计}  
\newcommand{\myTitleEN}{Multi-frequency digital up converter design} 

% 在这里填写你的相关信息
\newcommand{\myDept}{信息与通信工程学院}
\newcommand{\myMajor}{通信工程}
\newcommand{\myName}{张文杰}
\newcommand{\myClass}{现代通信技术}
\newcommand{\myMentor}{杜伟韬}
\newcommand{\studentID}{2021211103022}
\newcommand{\disdate}{2025年3月27日}

\begin{document}

\coverpage
\statement
% 中文摘要
\begin{abstract}
% 写摘要
随着 3G 时代的来临,移动通信技术正从提供简单的话音业务、低速数据业务向提
供话音业务和可变带宽的多媒体数据业务相结合的方向发展。天线,作为移动通信无线
链路中最重要的部件,也因此成为研究重点,受到了越来越广泛的关注。

\keywords {数字信号处理,音频信号频谱,matlab,fft,FPGA}
\end{abstract}

% 英文摘要
\begin{abstractEN}
% 写摘要 英文
With the coming of the era named „Third Generation‟, the mobile communication 
technologies are changing its direction from providing simple services of voice and low speed 
data transmission to providing combination services of voice and multi-media data 
transmission with variable band-width. Antennas, as the most important part in the radio link 
of mobile communication, become an important research topic, and attract more and more 
attentions.

\keywordsEN {dsp,spectral analysis,ff}       % 英文关键词
\end{abstractEN}

% 目录(自动生成)
\contentpage

\section{绪论} % 一级标题,可以用自己的章节标题去替换
\subsection{研究背景}
% 写背景

\subsection{研究现状}
% 写现状
三十四\cite{1018047768.nh}
\subsection{研究动机}
% 写动机

\subsection{研究目的与意义}
% 写目的写意义

\subsection{研究内容与结构安排}
% 写内容写安排



\section{相关技术}
% 写相关工作


\subsection{相关技术1}


\subsection{相关技术2}


\subsection{相关技术3}




\section{算法设计}
\label{sec:mothodology}

% 写模型
\section{实验结果与分析}
\label{sec:experiment}
% 写实验


\section{总结与展望}
\label{sec:conclusion}

\subsection{总结}
% 写总结

\subsection{未来工作}
% 写展望






% 参考文献数据库(所有需要引用的参考文献写入references.bib文件中)
\begin{references}
    \bibliography{references.bib}
\end{references}

\nonumsection{后记}  % 如果不需要,请将此部分内容做成注释或删除即可
% 尊敬的评阅老师感谢您能提出宝贵的建议。
\end{document}
